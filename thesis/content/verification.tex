\chapter{Overenie riešenia}

Keďže sme natrénovali viacero modelov, ako prvé sme ich porovnali na našej implementácii metódy RISE pre 3D snímky.
Následne sme vybrali z nich najlepší model, na ktorom sme vyskúšali rôzne nastavenia metódy RISEI, pretože skúšanie rôznych parametrov metódy RISEI je časovo náročné.

\section{Experimenty}

\subsection{Porovnanie metódy RISE na natrénovaných modeloch}

V tomto experiemnte sme porovnali metódu RISE na nami natrénovaných modeloch. Použili sme nami implementovaný metódu RISEI pričom sme nastavili jej parametre tak, aby fungovala ako metóda RISE.
Parametre sme nastavili nasledovne: $s = 8$, $p = 1/3$, $b1 = 0$, $b2 = 1$ (opis parametrov sa nachádza v tabuľke \ref{tab:rise_params}). Dosiahli sme ... TODO:s

\begin{table}[]
    \centering
    \begin{tabular}{l|r|r|r|}
        \cline{2-4}
                                        & \multicolumn{1}{c|}{3D CNN} & \multicolumn{1}{c|}{3D ResNet} & \multicolumn{1}{c|}{2D ResNet} \\ \hline
        \multicolumn{1}{|l|}{Insertion} & 0.00                        & 0.00                           & 0.00                           \\
        \multicolumn{1}{|l|}{Deletion}  & 0.00                        & 0.00                           & 0.00                           \\ \hline
    \end{tabular}
    \caption{Porovnanie metódy RISE na natrénovaných modeloch. Tepelné mapy sme generovali pre 25 náhodných snímkov z testovacej sady. Na tepelných mapách sme vyhodnocovali metriky insertion a deletion.}
    \label{tab:model_training_results}
\end{table}

\subsection{RISE vs RISEI}

\section{Záver}

Podarilo sa nám dosiahnuť ...
Vyhodnocovali sme zatiaľ len kvalitu tepelných máp a aj to na pomerne malej vzorke, taktiež sme nebrali do úvahy početnosť tried (AD a CN) v tejto vzorke (v ďaľších experimentoch by mali byť tieto triedy vyvážené).
Na generovanie tepelných máp sme použili 1000 masiek (keďže autori metódy RISE v experimentoch použili podobný počet), avšak my máme iný typ dát, preto je vhodné vyskúšať rôzne počty, a nájsť vhodný počet pre náš problém.

Keďže sú masky pri generovaní tepelných máp náhodné, je možné, že pre jeden MRI snímok metóda vygeneruje rôzne tepelné masky. V ďaľších experimentoch by sme mali vyhodnocovať aj konzistenciu tepelných máp - tj. ako veľmi sa líšia medzi rôznymi použitiami metódy na tom istom snímku.
Predpokladáme, že väčší počet použitých másk bude viesť ku konzistentnejším tepelným mapám. Takýmto meraním môžeme dospieť k optimálnemu počtu masiek, ktorý je potrebný na generovanie tepelnej mapy.

Taktiež je potrebné vyhodnotiť správnosť tepelných máp vzhľadom na segmentačné masky, tak ako sme uviedli v návrhu riešenia (Sekcia \ref{sec:heat_maps_and_model_segmentation_masks}). 
Ďalej je potrebné porovnať navrhovanú metódy s inými existujúcimi metódami, napr. LRP alebo analýza senzitivity.


% TODO: we need to compare our method with other methods like LRP or other oclussion methods
% TODO: evaluate the consistency of heatmaps, the method several times for the same image and find out if heatmaps are consistent - this way we can find also optimal number of masks
% TODO: compare with segmentation masks, if more heat is in important areas
