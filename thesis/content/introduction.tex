\chapter{Úvod}

% Uvod
% Spomenut neuronove siete, preco sa pouzivaju, preco ich ideme pouzit.
% Co robime, preco a ako.

Umelá inteligencia sa už dávno stala súčasťou nášho každodenného života. Pri-chádzame s ňou do kontaktu neustále, keď odomykáme telefón vlastnou tvárou alebo keď pomocou prekladača prekladáme text to iného jazyka. 
% spomenut nieco ako smartfon, vyhladavanie, fotak (ai camera), prekladanie?
Jej využitie je tiež rozšírené v oblasti medicíny, kde má potenciál zachraňovať životy. Využíva sa pri výrobe liekov, monitorovaní zdravia, analýze zdravotných plánov, chirurgických zákrokoch a aj pri odhaľovaní chorôb \cite{amisha2019overview}.
Práve pri odhaľovaní chorôb sa častokrát využívajú hlboké neurónové siete, a to napríklad pri detekcii rakoviny kože, rakoviny pľúc alebo Alzheimerovej choroby z obrazových dát.
% Dat zdroje k jednotlivym chorobam?
% Lung Cancer - https://www.sciencedirect.com/science/article/pii/S0169260713003532
% Skin Cancer - https://www.nature.com/articles/nature21056
% Alzheimer Disease - https://link.springer.com/article/10.1007/s10916-018-0932-7

Neurónovým sieťam sa už podarilo dosiahnuť také dobré výsledky, že sú porovnateľné s expertmi v medicínskej oblasti.
% "performance on par with all tested experts across both tasks, demonstrating an artificial intelligence capable of classifying skin cancer with a level of competence comparable to dermatologists"
% https://www.ncbi.nlm.nih.gov/pubmed/28117445
Ich problémom však je, že sa správajú ako "čierna skrinka", čo v oblasti medicíny nie je žiadúce. Preto je nevyhnutné, aby boli rozhodnutia neurónovej siete interpretovateľné a pacient s lekárom vedeli, na základe čoho sa neúronová sieť rozhodla. Lekári by si mali svoje rozhodnutia vedieť obhájiť. Aby sa teda neurónové siete mohli stať bežným pomocníkom lekárov, je vysvetliteľnosť ich rozhodnutí dôležitá.