\chapter{Zhodnotenie}

V našej práci sme sa venovali uplatneniu interpretovateľnosti a vysvetliteľnosti neurónových sietí pri vyhodnocovaní medicínskych obrazových dát. 

Skúmanú doménu a problém sme si naštudovali, a zistené informácie uviedli v analýze práce. Na základe toho sme navrhli modifikáciu existujúcej metódy na vyhodnocovanie rozhodnutí neurónových sietí. Výhodou navrhnutej metódy je, že nemusí poznať použitý model, a tak je ju možné použiť aj pri komplikovanejších modeloch (napr. kombinácia neurónovej siete a inej metódy strojového učenia, model so špecifickým predspracovaním do vektoru čŕt a pod.). Zároveň sme narvhli spôsob jej overenia - vyhodnotenie a porovnanie výsledkov.

Na základe návrhu sme naimplementovali modifikáciu existujúcej metódy RISE s dokreslením pre 3D volumetrické dáta. Vytvorenú metódy sme overlili v niekoľkých experimentoch na nami natrénovanom modeli. Vytvorená metóda disponuje viacerími parametrami ovplyvňujúcimi jej správanie (napr. hodnota prekrytia, miera prekrytia), preto sme v experimentoch overovali zvolené kombinácie parametrom, pričom u jednej dvojice parametrov sme prehľadávali mriežku parametrov. Keďže metóda používa na vytváranie tepelných máp masky s náhodným prekrytím, overili sme vplyv počtu vygenerovaných masiek na stabilitu tepelných máp. Metóda sa u vyššieho počtu vygenerovaných masiek ukázala ako stabilná.

Metódu sme porovnali s inými existujúcimi metódami - GradCAM, Guided Backprop a Guided GradCAM. Metóda v sledovaných metrikách dosiahla horšie výsledky ako metóda Guided Backprop, pričom dosiahla porovnateľné výsledky s metódou GradCAM. Kombináciou s Guided Backprop sme vytvorili metódu Guided RISE, ktorá dosiahla výsledky blízke Guided GradCAM a v niektorých ohľadoch takmer rovnaké s Guided Backprop. Použitie dokreslenia ako prekrytia sa neukázalo ako vhodný prístup v doméne rádiologických obrazových dát. 

Výsledné tepelné mapy sme vizualizovali. Tepelné mapy z metódy Guided Backprop naznačujú, že sa model nerozhoduje na základe relevantných častí mozgu. V ďaľších experiemntoch by bolo vhodné použiť lepší model.

\subsection{Zhodnotenie cieľov práce}

\paragraph{Vytvorenie novej, alebo vylepšenie existujúcej metódy pre vysvetľovanie rozhodnutí neurónových sietí (Sekcia \ref{sec:goals_1})}

Vytvorili sme modifikáciu už existujúcej metódy RISE, do ktorej sme priniesli niekoľko funkcionálnych vylepšení - podporu pre 3D volumetrické dáta a možnosť nastavenia rôznych hodnôt prepryvu (vrátane dokreslením). Ukázalo sa, že rôzne hodnoty prekryvu majú vplyv na kvalitu tepelných máp. V doméne rádiologických dát je vhodný prekryv inou hodnotou ako navrhli autori metódy RISE (Sekcia \ref{sec:verification_experiments_results}).

\paragraph{Využitie vytvorenej metódy na určenie miery správnosti modelu neurónovej siete detegujúcej Alzheimerovu chorobu (Sekcia \ref{sec:goals_2})}

Vytvorenú metódu sme využili na určenie správnosti modelu tak, že vytvorené tepelné mapy sme vyhodnocovali na základe nami zadefinovaných metrík (Sekcia \ref{sec:verification_experiments_metrics}). Tieto metriky využívali segmentačné masky, ktoré overovali, či vytvorená tepelná mapa dáva zmysel z pohľadu anatómie mozgu.

\subsection{Limitácie}

Oproti metódam ako je GradCAM alebo Guided Backprop vytvorená metóda vyžaduje viac času na vytvorenie tepelnej mapy. Ten je variabilný a závisí od počtu masiek, veľkosti pamäte, počtu jadier a GPU. Viac výpočtových zdrojov umožňuje generovanie masiek paralelizovať a generovať vo väčších dávkach.

\subsection{Budúca práca}

Vytvorená metóda bola (detailnejšie) overená iba na jednom modely. Pre jej dôkladnejšie overenie by bolo vhodné ju otestovať na modeloch a architektúrach s rôznymi úspešnosťami (najmä takými, ktoré sa blížia k 100\%). Zároveň sme identifikovali, že natrénovaný model sa pravdepodobne rozhodoval na základe lebky, z ktorej sa Alzheimerova choroba určiť nedá. Preto by lebka mohla byť v predspracovaní odstránená.

Ďalej navrhujeme:

\begin{itemize}
    \item Preskúmať vplyv počtu masiek na stabilitu tepelných máp, pretože to môže zredukovať ich potrebný počet a celkový čas generovania tepelných máp.
    \item Porovnať metódu s inými perturbačnými metódami, keďže sa navrhnutá metóda medzi ne zaraďuje. Navrhujeme porovnanie tepelných máp z hľadiska rýchlosti ich vytvárania a ich správnosti.
\end{itemize}
