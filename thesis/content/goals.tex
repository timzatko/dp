\chapter{Ciele práce \label{sec:goals}}

Vychádzajúc zo zadania projektu a na základe poznatkov nadobudnutých z analýzy domény a problému, sme si stanovili nasledovné ciele.

\section{Vytvorenie novej, alebo vylepšenie existujúcej metódy pre vysvetľovanie rozhodnutí neurónových sietí \label{sec:goals_1}} 

Existujú rôzne metódy pre vysvetľovanie rozhodnutí neurónových sietí. Či už sú to tzv. white-box metódy (ako napríklad LRP) alebo tzv. black-box metódy, ktoré je možné použiť na ľubovoľný typ modelu. Žiadna z týchto metód nie je dokonalá (každá má svoje plusy a mínusy v rôznych aspektoch) a je tu teda priestor na vytvorenie novej (lepšej) alebo vylepšenie existujúcej metódy. V prípade vylepšenia existujúcej metódy je nutné túto metódu porovnať najmä s vylepšovanou metódou a následne s inými metódami. Cieľom je teda vytvoriť novú metódu, ktorá vytvára presnejšie vysvetlenia ako iné metódy, alebo vylepšiť existujúcu metódu, ktorá vytvára presnejšie vysvetlenia ako metóda, z ktorej vychádza.

\section{Využitie vytvorenej metódy na určenie miery správnosti modelu neurónovej siete detegujúcej Alzheimerovu chorobu \label{sec:goals_2}}

Pri neurónových sieťach detekujúcich Alzheimerovu chorobu je dôležité, aby sa naučili klasifikovať pacientov na základe relevatných čŕt z rádiologických snímkov. Práve preto je potrebné určiť mieru správnosti modelu podľa toho či sa model rozhoduje práve na základe týchto čŕt a nie iných. Na to sa využívajú metódy na vysvetľovanie rozhodnutí neurónových sietí, v tomto prípade sa použije novovytvorená metóda. Cieľom je teda určiť správnosť modelu detegujúceho Alzheimerovu chorobu pomocou vytvorenej metódy pre vysvetľovanie rozhodnutí neurónovej siete.
