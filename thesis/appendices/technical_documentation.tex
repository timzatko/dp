\chapter{Technická dokumentácia \label{cha:technical_documentation}}

\setcounter{page}{1}

Metóda RISEI je implementovaná v jazyku Python, rovnako ako aj jej vyhodnotenie
a porovnanie s ostatnými metódami.

\section{Príprava vývojového prostredia}

Na správu python-ovských balíkov a vývojového prostredia je použitá \href{https://docs.conda.io/projects/conda/en/latest/index.html}{conda}, ktorú je nutné nainštalovať. Condu je možné nainštalovať cez distribúciu \href{https://www.anaconda.com/anaconda/}{Anaconda} (Anaconda obsahuje grafické rozhranie a množstvo nástrojov/programov) alebo menšiu distribúciu \href{https://docs.conda.io/en/latest/miniconda.html}{Miniconda}. V prípade, že potrebujete šetriť miesto na disku, odporúčam menšiu distribúciu Miniconda.

Po inštalácii condy zadajte nasledovný príkaz v koreňovom adresári repozitára. Tento príkaz vytvorí nové conda prostredie v ktorom nainštaluje potrebné python balíky.

\begin{lstlisting}[style=BashInputStyle]
    $ conda env create -f environment.yml
\end{lstlisting}

Následne pre aktiváciu conda prostredia zadajte nasledovný príkaz.

\begin{lstlisting}[style=BashInputStyle]
    $ conda activate dp-timzatko
\end{lstlisting}

Teraz je možné používať shell, v ktorom bolo aktivované conda prostredie \textit{dp-timzatko}, na spúšťanie Python skriptov a Jupyter notebookov.

Následne spustite Jupyter notebook klienta.

\begin{lstlisting}[style=BashInputStyle]
    $ jupyter-notebook
\end{lstlisting}

Teraz je možné, prezerať, spúšťať a upravovať jupyter notebooky v repozitári.

\section{Závislosti (použité knižnice)}

Na implementáciu riešenia sme použili nasledovné Python knižnice (uvádzame len tie najvýznamnejšie). Kompletný zoznam sa nachádza v súbore \textit{[REPOZITÁR]/environment.yml}.

\begin{itemize}
    \item numpy - na prácu s vektormi, maticami a matematickými operáciami nad nimi.
    \item pandas - na vytváranie, a ukladanie tabuľkami.
    \item seaborn - vykresľovanie grafov.
    \item matplotlib - vykresľovanie rádiologických snímkov, tepelných mám a segmentačných masiek.
    \item opencv - na dokreslenie v RISEI.
    \item tensorflow (v2.3.1), tensorboad - implementácia, trénovanie, evaluácia modelu na predikciu alzheimerovej choroby.
    \item torch, torchvision - evaluácia modelu na predikciu alzheimerovej choroby. Pytorch je potrebný, pretože je závislosťou knižnice \textit{captum}, ktorú používame na vytváranie tepelných masiek existujúcimi metódami (GradCAM a pod.).
    \item SimpleITK - načítanie volumetrických dát z disku.
    \item scikit-image - práca s vizuálnymi dátami (augmentácie, zmena veľkosti).
\end{itemize}

\section{Technické riešenie}

Implementácia riešenia (RISEI, model, evaluácia atď.) sa nachádza v adresári \textit{[REPOZITÁR]/src}. Funkcionalita z tohto adresára je následne importovaná jupyter notebookmi v adresári \textit{[REPOZITÁR]/conda\_notebooks}. Každý z notebookov má iný účel - trénovanie modelu, vyhodnotenie metódy RISEI, porovnanie metód a pod (detailný opis k týmto notebookom sa nachádza v prílohe \ref{cha:cdrom} Opis digitálnej časti práce).

\section{Moduly}

Adresár \textit{[REPOZITÁR]/src} obashuje nasledovné Python moduly, ktoré majú uvedené zodpovednosti.

\begin{itemize}
    \item \textbf{src.risei} - implementácia metódy RISEI (exportuje triedu RISEI) - generovanie masiek.
    \item \textbf{src.model} - obsahuje pomocné funkcie na prácu s tensorflow modelom (načítanie checkpointu atď.). 
    \item \textbf{src.model.cnn\_3D} - implementácia 3D konvolučnej neurónovej siete v tensorflow-e.
    \item \textbf{src.model.res\_net} - implementácia 3D a 2D siete ResNet v tensorflow-e.
    \item \textbf{src.model.compile\_model} - kompilácia tensorflow modelu a nastavenie metrík, optimizéru, chybovej funkcie a predvolených nastavení.
    \item \textbf{src.model.create\_model} - vytvorenie modelu.
    \item \textbf{src.model.training} - spustenie trénovania modelu, vrátane vytvorenie tensorflow datasetu, nastavenia augmentácií, pripojenia k tensorboardu a pod. na základe vstupných parametrov.
    \item \textbf{src.model.mri\_tensorboad\_callback} - výpis rádiologických snímkov po epochách/iteráciách do tensorboard-u pri trénovaní.
    \item \textbf{src.model.evaluation} - vyhodnotenie modelu (matica zmätenia, klasifikačné metriky) a priebehu jeho trénovania.
    \item \textbf{src.model.torch.cnn\_3D} - implementácia 3D konvolučnej neurónovej siete v pytorch-y.
    \item \textbf{src.data} - práca s volumetrickými dátami, konverzia tensorflow sequence do numpy a opačne.
    \item \textbf{src.data.description} - popisné štatistiky o dátovej sade.
    \item \textbf{src.data.augmentations} - augmentácie.
    \item \textbf{src.data.mri\_sequence} - načítanie MRI snímkov, štítkov a segementačných masiek z disku. Zmena veľkosti snímkov, orezanie snímkov, štandardizácia dát, rozdelenie dát do dávok.
    \item \textbf{src.data.train\_test\_split} - rozdelenie dátovej sady na trénovaciua, testovaciu a validačnú.
    \item \textbf{src.data.selector} - výber záznamov z dátovej sadny na základe triedy AD/CN a správnosti klasifikácie modelom.
    \item \textbf{src.data.evaluation.segmentation\_masks} - evaluácia tepelných máp podľa segmentačných masiek.
    \item \textbf{src.heatmaps} - generovanie tepelných máp.
    \item \textbf{src.heatmaps.evaluation} - evaluácia kvality tepelnej mapy podľa metrík \textit{insertion} a \textit{deletion}, perzistencia histórie - tepelných máp a ich evaluácie, načítanie histórie evaluácie, vizualizácie v diagramoch. 
\end{itemize}

Funkcie a triedy v moduloch sú v primeranom rozashu dokumentované pomocou komentárov. Ďalej bližšie opíšeme najviac dôležitý modul implementujúci navrhnutnú metódu \textit{RISEI}.

\subsection{Modul: src.risei}

Tento modul poskytuje triedu RISEI ktorá slúži na generovanie masiek, z ktorých sa vytvárajú tepelné mapy.

\subsubsection{Trieda: RISEI}

Trieda RISEI slúži na generovanie masiek.

\begin{lstlisting}
    class src.risei.RISEI(input_size, s=8, p1=0.5, b1=0.8, b2=0.5,
        b2_value=0,
        in_paint='2d',
        in_paint_radius=20,
        in_paint_algorithm=cv2.INPAINT_NS,
        in_paint_blending=True,
        in_paint_2d_to_3d=False,
        processes=4,
        debug=False,
    )
\end{lstlisting}

\paragraph{Parametre}

\begin{itemize}
    \item \textbf{s} - veľkosť mriežky, z ktorej sa vytvára binárna maska.
    \item \textbf{p1} - pravdepodobnosť, že pixel v mriežke bude biely.
    \item \textbf{b1} - miera prekryvu medzi pôvodným obrázkom a dokreslením. Ak je \textit{0} tak sa dokreslenie vôbec nevykoná.
    \item \textbf{b2} - miera prekryvu medzi pôvodným obrázkom s dokreslením a ''čiernou'' maskou. 
    \item \textbf{b2\_value} - hodnoty v ''čiernej'' maske.
    \item \textbf{in\_paint} - typ dokreslenia. Môže byť \textit{2d} alebo \textit{3d}. V prípade \textit{2d} je dokreslenie realizované iba v prvej dimenzii.
    \item \textbf{in\_paint\_radius} - rádius dokreslenia (posúva sa ďalej do knižnice \textit{opencv}).
    \item \textbf{in\_paint\_algorithm} - algoritmus dokreslenia, môže byť \textit{cv2.INPAINT\_NS} alebo \textit{cv2.INPAINT\_TELEA}.
    \item \textbf{in\_paint\_blending} - ak \textit{True} dokreslenie bude prekryté s pôvodným snímkom podľa interpolovanej čiernej masky (tak nevzniknú žiadne ostré hrany).
    \item \textbf{processes} - počet procesov v ktorých sa budú generovať masky.
    \item \textbf{debug} - ak \textit{True} budú do pamäte ukladané medzivýsledky z generovania masiek (binárna maska, interpolovaná maska atď.).
\end{itemize}

\paragraph{Metódy}

Metódy triedy RISEI.

\vspace{8pt}
\begin{lstlisting}
    generate_masks(n, image, log=True, seed=None)
\end{lstlisting}

Metóda vygeneruje masky k poskytnutej snímke ($image$).

Parametre:
\begin{itemize}
    \item \textbf{n} - počet masiek na vygenerovanie,
    \item \textbf{image} - 3D snímka \textit{(x, y, z)} (k nej budú vygenerované masky),
    \item \textbf{log} - ak \textit{True} na štandardnom výstupe bude zobrazený aktuálny stav generovania masiek,
    \item \textbf{seed} - seed pre náhodu.
\end{itemize}

\vspace{32pt}
\begin{lstlisting}
    show_from_last_run(i, z, figsize=(12, 8), dim=0):
\end{lstlisting}

Parametre:
\begin{itemize}
    \item \textbf{i} - i-ta maska na zobrazenie,
    \item \textbf{z},
    \item \textbf{figsize} - veľkosť vykresleného diagarmu,
    \item \textbf{dim} - reprezentuje rozmer - 0, 1, 2.
\end{itemize}

\subsection{Modul: src.heatmaps.evaluation}

Temto modul poskytuje generovanie a evaluáciu kvality tepelnej mapy podľa metrík \textit{insertion} a \textit{deletion}, perzistenciu histórie - tepelných máp a ich evaluácie, načítanie histórie a vizualizácie v diagramoch.

\subsubsection{Trieda: HeatmapEvaluationV3}

Táto trieda vytvorí a vyhodnotí tepelné mapy, následne vráti históriu evaluácie.

\begin{lstlisting}
    class src.risei.HeatmapEvaluationV3(
        predict_fn,
        heatmap_fn,
        sequence,
        evaluation_step_size=1000,
        evaluation_max_steps=-1,
        evaluation_batch_size=32)
\end{lstlisting}

\paragraph{Parametre}

\begin{itemize}
    \item \textbf{predict\_fn} - funkcia \textit{def predict\_fn(batch\_x)}, ktorá pre dávku snímkov vráti pravdepodobnosti predikovaných pre jednotlivé triedy.
    \item \textbf{heatmap\_fn} - funkcia \textit{heatmap\_fn(image\_x, image\_y, evaluation\_idx, seed, log}, ktorá vráti tepelnú mapu pre snímku \textit{image\_x}.
    \item \textbf{sequence} - sekvencia snímok - musí byť \textit{src.data.MRISequence}.
    \item \textbf{evaluation\_step\_size} - koľko voxelov bude vložených v jednom kroku pri evaluácii tepelnej mapy. 
    \item \textbf{evaluation\_batch\_size} - koľko snímkov vkladať pri evaluácii do modelu.
\end{itemize}

\paragraph{Metódy}

Metódy triedy HeatmapEvaluationV3.

\vspace{8pt}
\begin{lstlisting}
    evaluate(log=False, verbose=0, seed=None)
\end{lstlisting}

Spustí generovanie a evaluáciu tepelných máp. Metóda vráti \textit{src.heatmaps.\-evaluation.\-Heatmap\-EvaluationHistory} s históriou evaluácie a s metrikami.

Parametre:
\begin{itemize}
    \item \textbf{log} - zapne alebo vypne výpisy.
    \item \textbf{verbose} - úroveň výpisov 0, 1, 2 (najmenej až najviac).
    \item \textbf{seed} - seed pre náhodu (posunutý do \textit{heatmap\_fn}).
\end{itemize}

\subsubsection{Trieda: HeatmapEvaluationHistory}

Táto uchováva históriu evaulácie tepelných máp. Umožňuje ich načítavať a ukladať na disk.

\begin{lstlisting}
    class src.risei.HeatmapEvaluationHistory(
        method, auc, arr_auc, arr_heatmap, arr_x, arr_y, arr_y_pred, arr_y_pred_heatmap,
                 arr_voxels, arr_max_voxels, arr_step_size)
\end{lstlisting}

Túto triedu inicializuje knižnica (v \textit{src.heatmaps.\-evaluation.\-HeatmapEvaluationV3}), nemal by ju inicializovať klientský kód.

\paragraph{Metódy}

Metódy triedy HeatmapEvaluationHistory.

\vspace{8pt}
\begin{lstlisting}
    save(path, filename)
\end{lstlisting}

Uloží históriu na disk.

Parametre:
\begin{itemize}
    \item \textbf{path} - cesta k súboru.
    \item \textbf{filename} - názov súboru.
\end{itemize}

\vspace{24pt}
\begin{lstlisting}
    @staticmethod
    def load(path, filename):
\end{lstlisting}

Načíta históriu z disku.

Parametre:
\begin{itemize}
    \item \textbf{path} - cesta k súboru.
    \item \textbf{filename} - názov súboru.
\end{itemize}

\vspace{24pt}
\begin{lstlisting}
    description(percentage=True, cls_index=None)
\end{lstlisting}

Vypíše popisné štatistiky (min, max, priemer, štandardná odchýlka) pre plochy pod krivkou jednotlivých evaluácií.

Parametre:
\begin{itemize}
    \item \textbf{percentage} - ak $False$ vráti AUC v absolútnych číslach vzhľadom na počet voxelov a nie hodnotu z intervalu $<0, 1>$.
    \item \textbf{cls\_index} - ak $None$ vráti štatistiky pre všetky pozorovanie, ak číslo, vráti štatistiky pre index danej triedy.
\end{itemize}

Táto trieda poskytuje aj ďaľšie metódy pre vizualizáciu plochy pod krivkou pre jednotlivé pozorovania a pod., popísali sme iba najdôležitejšie z nich.

\paragraph{Atribúty}

\begin{itemize}
    \item \textbf{arr\_heatmap} - vygenerované tepelné mapy pre všetky snímky.
    \item \textbf{arr\_x} - snímky.
    \item \textbf{arr\_y} - skutočne triedy ku snímkam.
    \item \textbf{arr\_y\_pred} - predikované triedy ku snímkam.
    \item \textbf{arr\_auc} - hodnoty AUC pre jednotlivé snímky.
\end{itemize}

\thispagestyle{empty}
\mbox{}
\newpage