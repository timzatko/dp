\chapter{Plán práce \label{cha:plan}}

\setcounter{page}{1}

\section{Letný semester - DP1}

V tomto semestri plánujem pracovať na analýze domény, návrhu metódy a jej implementácii.

\section{Zimný semester - DP2}

    % \subsection{Plán práce}
    
    V tomto semestri plánujem pracovať na implementácii navrhnutej metódy, ktorú budem overovať v experimentoch a postupne vylepšovať. V tomto semestri plánujem:
    
    \begin{itemize}
        \item Natrénovať model na detekciu Alzheimerovej choroby z MRI snímkov
        \item Implementovať navrhnutú metódu
        \item Experimentovať s hyper-parametrami navrhnutej metódy
        \item Skúmať dosiahnuté výsledky, hľadať príčiny a možné vylepšenia
        \item Priebežne písať prácu -- implementáciu a dosiahnuté výsledky
    \end{itemize}

    \subsection{Vyjadrenie k plneniu plánu}

    V tomto semestri sa nám podarilo splniť všetky stanovené ciele. Natrénovali sme niekoľko modelov detekujúcich Alzheimerovu chorobu z MRI snímkov. Čo sa týka úspešnosti týchto modelov, bohužiaľ sa nám nepodarilo dosiahnuť tak dobré výsledky ako u iných prác. Avšak, naším cieľom nie je natrénovať najlepší model, takže táto úspešnosť vyzerá byť zatiaľ pre nás postačujúca. 

    Metódu sme implementovali, tak, ako sme ju navrhli, pričom sme pridali vylepšenia ako multiprocessing - paralelné generovanie masiek.

    S hyper-parametrami navrhnutej metódy sme experimentovali (ale nie so všetkými, pretože ich je veľa), avšak sme nerobili žiadne prehľadávanie optimálnych parametrov.

    Dosiahnuté výsledky sme skúmali a diskutovali ich v závere overenia riešenia pričom sme navrhli ďalšie kroky.


\section{Letný semester - DP3}

    % \subsection{Plán práce}
    
    V tomto semestri budem pracovať na finalizácii tejto práce, navrhnutú metódu plánujem už iba vylepšovať a pracovať na záverečnom dokumente. V tomto semestri plánujem:
    
    \begin{itemize}
        \item Písať prácu a jej jednotlivé časti - implementácia, technická dokumentácia, dosiahnuté výsledky, záver
        \item Vykonať úpravy v navrhnutej metóde na základe doterajších výsledkov experimentov
        \item Vyhodnotiť stabilitu tepelných máp
        \item Optimalizovať vstupné parametre do RISEI metódy
        \item Porovnať navrhnutú metódy s existujúcimi metódami
        \item Vyhodnotiť a porovnať vykonané experimenty
        \item Odovzdať prácu
    \end{itemize}
    
    \subsection{Vyjadrenie k plneniu plánu}
    
    Plán práce sa nám v tomto semestri podarilo dodržať.
    Z experimentov nevzišli žiadne potrebné úpravy metódy, preto žiadne úpravy metódy neboli v tomto semestri vykonané. Taktiež sme vyhodnotili stabilitu tepelných máp. Rovnako sme hľadali aj optimálnu kombináciu vstupných parametrov metódy RISEI tak, že sme si vytvorili možné kombinácie parametrov, ktoré sme vyhodnotili. Neoptimalizovale sme ale všetky vstupné parametre, pri niektorých sme uznali, že to u nich nedáva zmysel. Vytvorenú metódu sme taktiež porovnali s existujúcimi metódami GradCAM, Guided Backprop a Guided GradCAM. Vykonali sme veľké množstvo experimentov (rôzne parametre RISEI, kombinácia RISEI a guided backprop atď.), ktoré sme vyhodnotili (napr. vyhodnotenie na segmentačných maskách) a porovnali.
