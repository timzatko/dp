% https://www.fiit.stuba.sk/buxus/docs/organizacia_studia/pokyny/ZP-clenenie-pokyny.pdf

\thispagestyle{empty}

\section*{Annotation}

\begin{minipage}[t]{1\columnwidth}%
Slovak University of Technology Bratislava 

Faculty of Informatics and Information Technologies

Degree Course: \myStudyProgramEng\\

Author: \myName

Diploma's Thesis: \myTitleEng

Supervisor: \mySupervisorEng

% TODO: Pri final odovzdani zmenit mesiac
2019, May
% 2018, December
\end{minipage}

\bigskip{}

The current impact of artificial intelligence on society is undeniable. It has already been used in various areas of our lives, whether it is in smartphones for unlocking via face recognition or, most recently, for controlling the use of protective masks when entering shops or groceries. Artificial intelligence is entering the field of medicine, where it has the potential to save lives. Thus, in order to be a reliable assistant to doctors for example in the diagnosis of the disease, it is necessary that its decisions can be explained.

In the field of medicine, the usage of neural networks is possible, because they can work very well with image data, and so they can be used, for example, in the diagnosis of Alzheimer's disease from radiological images. However, their problem is that they behave like a "black box" which prevents them from getting into common practice.

In this work, we proposed a novel method of interpreting neural networks, we proposed a process of comparison with existing approaches and verification in explaining the neural network decisions detecting Alzheimer disease from MRI images.

\newpage
\thispagestyle{empty}
\mbox{}
\newpage