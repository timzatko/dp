% https://www.fiit.stuba.sk/buxus/docs/organizacia_studia/pokyny/ZP-clenenie-pokyny.pdf

\thispagestyle{empty}

\section*{Annotation}

\begin{minipage}[t]{1\columnwidth}%
Slovak University of Technology Bratislava 

Faculty of Informatics and Information Technologies

Degree Course: \myStudyProgramEng\\

Author: \myName

Diploma's Thesis: \myTitleEng

Supervisor: \mySupervisorEng

2021, May
\end{minipage}

\bigskip{}

The current impact of artificial intelligence on society is undeniable. It has already been used in various areas of our lives, whether it is in smartphones for unlocking via face recognition or, most recently, for controlling the use of protective masks when entering shops or groceries. Artificial intelligence is entering the field of medicine, where it has the potential to save lives. Thus, in order to be a reliable assistant to doctors for example in the diagnosis of the disease, it is necessary that its decisions can be explained.

Since neural networks perform well on image data, they can be used in the field of medicine, for example, in the diagnosis of Alzheimer's disease from radiological images. However, their problem is that they behave like a "black box" which prevents their adoption into common practice.

In this work, we designed and implemented a novel method for explaining neural network decisions, which is based on other existing method - RISE. The created method is suitable for image data, for which it creates heatmaps explaining the decisions of the model.

We have validated the method on a neural network detecting Alzheimer's disease from MRI images and compared it to the original, and other existing methods. Although the new method achieved better results than the original, it did not prove to be suitable for radiological data.

\newpage
\thispagestyle{empty}
\mbox{}
\newpage