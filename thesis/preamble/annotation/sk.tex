% https://www.fiit.stuba.sk/buxus/docs/organizacia_studia/pokyny/ZP-clenenie-pokyny.pdf

\thispagestyle{empty}
\section*{Anotácia}

\begin{minipage}[t]{1\columnwidth}%
Slovenská technická univerzita v Bratislave

Fakulta informatiky a informačných technológií

Študijný program: \myStudyProgram\\

Autor: \myName

Diplomová práca: \myTitle

Vedúci diplomového projektu: \mySupervisor

máj 2021
\end{minipage}

\bigskip{}

Súčasný vplyv umelej inteligencie na spoločnosť je nespochybniteľný. Využitie si už našla v rôznych oblastiach našich životov či už je to v smartfónoch pri odomykaní tvárou alebo najnovšie pri kontrole používania ochranného rúška pri vstupe do obchodov. Umelá inteligencia sa postupne dostáva do oblasti medicíny, kde má potenciál zachraňovať životy. Aby sa stala spoľahlivým pomocníkom doktorov pri diagnóze ochorení, je nevyhnutné, aby jej rozhodnutia bolo možné vysvetliť. 

V oblasti medicíny je možné použitie neurónových sietí, pretože dokážu veľmi dobre pracovať s obrazovými dátami, a tak sa dajú využiť napríklad pri diagnóze Alzheimerovej choroby z rádiologických snímok. Ich problémom však je, že sa správajú ako "čierne skrinky" čo bráni ich použitiu v bežnej praxi.

V tejto práci sme navrhli a implementovali novú metódu vysvetľovania rozhodnutí neurónových sietí, ktorá vychádza z už existujúcej metódy - RISE. Vytvorená metóda je vhodná pre obrazové dáta, ku ktorým vytvára tepelné mapy vysvetľujúce rozhodnutia modelu.

Metódu sme overili na neurónovej sieti deketujúcej Alzheimerovu chorobu z MRI snímok a porovnali ju voči pôvodnej metóde a iným existujúcim metódam. Nová metóda síce dosiahla oproti pôvodnej lepšie výsledky, avšak sa neukázala byť vhodná pre rádiologické dáta.

\newpage
\thispagestyle{empty}
\mbox{}
\newpage