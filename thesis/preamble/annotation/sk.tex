% https://www.fiit.stuba.sk/buxus/docs/organizacia_studia/pokyny/ZP-clenenie-pokyny.pdf

\thispagestyle{empty}
\section*{Anotácia}

\begin{minipage}[t]{1\columnwidth}%
Slovenská technická univerzita v Bratislave

Fakulta informatiky a informačných technológií

Študijný program: \myStudyProgram\\

Autor: \myName

Diplomová práca: \myTitle

Vedúci diplomového projektu: \mySupervisor

% TODO: DP3 pri final odovzdani zmenit mesiac
január 2021
\end{minipage}

\bigskip{}

Súčasný vplyv umelej inteligencie na spoločnosť je nespochybniteľný. Využitie si už našla v rôznych oblastiach našich životov či už je to v smartfónoch pri odomykaní tvárou alebo najnovšie pri kontrole používania ochranného rúška pri vstupe do obchodov. Umelá inteligencia sa postupne dostáva do oblasti medicíny, kde má potenciál zachraňovať životy. Aby, teda mohla byť spoľahlivým pomocníkom doktorov pri diagnóze ochorení je nevyhnutné, aby jej rozhodnutie bolo možné vysvetliť. 

V oblasti medicíny je možné použitie neurónových sietí, pretože dokážu veľmi dobre pracovať s obrazovými dátami, a tak sa dajú využiť napríklad pri diagnóze Alzheimerovej choroby z rádiologických snímkov. Ich problémom však je, že sa správajú ako "čierna skrinka" čo bráni v tom, aby sa dostali do bežnej praxe.

V tejto práci sme navrhli nový spôsob interpretovania neurónových sietí, navrhli sme spôsob porovnania s existujúcimi prístupmi a overenia pri vysvetľovaní rozhodnutí neurónovej siete deketujúceh Alzheimerovu chorobu z MRI snímkov.

\newpage
\thispagestyle{empty}
\mbox{}
\newpage